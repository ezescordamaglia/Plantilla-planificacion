\documentclass[
11pt, % The default document font size, options: 10pt, 11pt, 12pt
%codirector, % Uncomment to add a codirector to the title page
]{charter} 


% El títulos de la memoria, se usa en la carátula y se puede usar el cualquier lugar del documento con el comando \ttitle
\titulo{Construcción de un modelo para predecir la mortalidad en pacientes en diálisis renal} 

% Nombre del posgrado, se usa en la carátula y se puede usar el cualquier lugar del documento con el comando \degreename
\posgrado{Carrera de Especialización en Inteligencia Artificial}

% Tu nombre, se puede usar el cualquier lugar del documento con el comando \authorname
\autor{Lic. Ezequiel Scordamaglia} 

% El nombre del director y co-director, se puede usar el cualquier lugar del documento con el comando \supname y \cosupname y \pertesupname y \pertecosupname
\director{Nombre del Director}
\pertenenciaDirector{pertenencia} 
% FIXME:NO IMPLEMENTADO EL CODIRECTOR ni su pertenencia
%\codirector{John Doe} % para que aparezca en la portada se debe descomentar la opción codirector en el documentclass
\pertenenciaCoDirector{FIUBA}

% Nombre del cliente, quien va a aprobar los resultados del proyecto, se puede usar con el comando \clientename y \empclientename
\cliente{Eugenio Bellia}
\empresaCliente{Grupo DUAM}

% Nombre y pertenencia de los jurados, se pueden usar el cualquier lugar del documento con el comando \jurunoname, \jurdosname y \jurtresname y \perteunoname, \pertedosname y \pertetresname.
\juradoUno{Nombre y Apellido (1)}
\pertenenciaJurUno{pertenencia (1)} 
\juradoDos{Nombre y Apellido (2)}
\pertenenciaJurDos{pertenencia (2)}
\juradoTres{Nombre y Apellido (3)}
\pertenenciaJurTres{pertenencia (3)}
 
\fechaINICIO{17 de octubre de 2023}		%Fecha de inicio de la cursada de GdP \fechaInicioName
\fechaFINALPlan{5 de diciembre de 2023} 	%Fecha de final de cursada de GdP
\fechaFINALTrabajo{7 de junio de 2024}	%Fecha de defensa pública del trabajo final


\begin{document}

\maketitle
\thispagestyle{empty}
\pagebreak


\thispagestyle{empty}
{\setlength{\parskip}{0pt}
\tableofcontents{}
}
\pagebreak


\section*{Registros de cambios}
\label{sec:registro}


\begin{table}[ht]
\label{tab:registro}
\centering
\begin{tabularx}{\linewidth}{@{}|c|X|c|@{}}
\hline
\rowcolor[HTML]{C0C0C0} 
Revisión & \multicolumn{1}{c|}{\cellcolor[HTML]{C0C0C0}Detalles de los cambios realizados} & Fecha      \\ \hline
0      & Creación del documento                                 &\fechaInicioName \\ \hline
1      & Se completa hasta el punto 4 inclusive                 & 26 de octubre de 2023 \\ \hline
2      & Se completa hasta el punto 9 inclusive					& 3 de noviembre de 2023 \\ \hline
3      & Se completa hasta el punto 12 inclusive                & 13 de noviembre de 2023 \\ \hline
4      & Se completa el plan	                                & 20 de noviembre de 2023 \\ \hline
\end{tabularx}
\end{table}

\pagebreak



\section*{Acta de constitución del proyecto}
\label{sec:acta}

\begin{flushright}
Buenos Aires, \fechaInicioName
\end{flushright}

\vspace{2cm}

Por medio de la presente se acuerda con el \authorname\hspace{1px} que su Trabajo Final de la \degreename\hspace{1px} se titulará ``\ttitle'', y consistirá en el desarrollo de un modelo predictivo, como prueba piloto, que prediga el nivel de riesgo de mortalidad de un paciente que se encuentre en tratamiento de diálisis renal. Este desarrollo se enmarca en la actualización tecnológica de la organización \empclientename , donde se busca aplicar inteligencia artificial y análisis de datos a nuevos desarrollos que ofrezcan valor agregado a los clientes. El mismo tendrá un presupuesto preliminar estimado de 600 h de trabajo y \$3.864.000, con fecha de inicio \fechaInicioName\hspace{1px} y fecha de presentación pública \fechaFinalName.

Se adjunta a esta acta la planificación inicial.

\vfill

% Esta parte se construye sola con la información que hayan cargado en el preámbulo del documento y no debe modificarla
\begin{table}[ht]
\centering
\begin{tabular}{ccc}
\begin{tabular}[c]{@{}c@{}}Dr. Ing. Ariel Lutenberg \\ Director posgrado FIUBA\end{tabular} & \hspace{2cm} & \begin{tabular}[c]{@{}c@{}}\clientename \\ \empclientename \end{tabular} \vspace{2.5cm} \\ 
\multicolumn{3}{c}{\begin{tabular}[c]{@{}c@{}} \supname \\ Director del Trabajo Final\end{tabular}} \vspace{2.5cm} \\
%\begin{tabular}[c]{@{}c@{}}\jurunoname \\ Jurado del Trabajo Final\end{tabular}     &  & \begin{tabular}[c]{@{}c@{}}\jurdosname\\ Jurado del Trabajo Final\end{tabular}  \vspace{2.5cm}  \\
%\multicolumn{3}{c}{\begin{tabular}[c]{@{}c@{}} \jurtresname\\ Jurado del Trabajo Final\end{tabular}} \vspace{.5cm}                                                                     
\end{tabular}
\end{table}




\section{1. Descripción técnica-conceptual del proyecto a realizar}
\label{sec:descripcion}

Este proyecto, planteado como trabajo final, se realizará para la organización Grupo DUAM, la cual desarrolla software a medida para distintos clientes.
El objetivo principal de este proyecto es incursionar en desarrollos que utilicen las nuevas tecnologías, como son la inteligencia artificial y el análisis de datos, con el fin de proporcionar soluciones innovadoras a problemas preexistentes.
En esta línea, se tomó la decisión de colaborar con uno de sus clientes, una empresa médica que opera centros de atención en todo el país, para crear un primer prototipo que haga uso de estas tecnologías y que, al mismo tiempo, les aporte un valor agregado. Ellos poseen un sistema de gestión de pacientes donde registran datos médicos asociados a cada paciente, como son estudios médicos, hospitalizaciones, medicamentos prescritos, y muchos otros. Es fundamental en la medicina conocer el estado de salud de los pacientes y evaluar los riesgos asociados a ellos. Esto le permite a los profesionales de la salud ajustar el tratamiento y los medicamentos que  prescriben. Pero predecir los riesgos que puede tener un paciente no es tarea sencilla. La complejidad de las condiciones de salud y la gran cantidad de datos clínicos disponibles hacen extremadamente difícil realizar una predicción de mortalidad de manera manual.
En este contexto, surge la idea de ofrecerles desarrollar un modelo predictivo de mortalidad que pueda aprovechar el poder de la inteligencia artificial y del análisis de datos, para realizar un pronóstico rápido y preciso del nivel de riesgo que tiene un paciente.

En este trabajo, se propone partir por la creación de un modelo predictivo de mortalidad que sirva de ejemplo y permita ir desarrollando toda la infraestructura necesaria para que el modelo quede disponible para los usuarios. Se utilizará una plataforma de gestión de modelos y se desarrollará una interfaz por servicio web y un proceso automatizado que solicite las predicciones. Para el desarrollo de este modelo predictivo puntual, se contará con información de miles de pacientes que fueron atendidos por esta empresa médica y sirven de referencia para el proceso de entrenamiento. Se deberán procesar los datos disponibles, entrenar varios modelos de \textit{machine learning} y \textit{deep learning}, evaluar sus métricas y seleccionar los candidatos que logren los mejores resultados.
El desarrollo de este tipo de modelos predictivos, administrados por una plataforma de gestión integral y accesibles por servicio web, permitirá adentrarse en un mercado en constante evolución que demanda soluciones innovadoras. En la figura \ref{fig:diagBloques} se muestra un diagrama de la solución propuesta para resolver el caso piloto.

\begin{figure}[htpb]
\centering 
\includegraphics[width=.85\textwidth]{./Figuras/DiagramaDeBloques.pdf}
\caption{Diagrama en bloques del sistema.}
\label{fig:diagBloques}
\end{figure}

Si bien parte del proyecto se limitará a dar solución a un problema puntual en el ámbito médico, el objetivo final es poder brindar soluciones a distintos clientes, incorporando los beneficios de las nuevas tecnologías.
El uso de la inteligencia artificial y el análisis de datos está en auge actualmente, y muchas industrias de distintos campos ya están logrando resultados exitosos. En la medicina ya se han utilizado para la predicción del desarrollo de enfermedades como la diabetes o el cáncer, y también para detectar formas extrañas en imágenes. 
Esto se debe a que los modelos utilizados tienen la capacidad de reconocer patrones en grandes conjuntos de datos.
Para poder generar predicciones es necesario utilizar datos históricos en el entrenamiento, donde los modelos descubren relaciones entre ellos. Luego, por métodos estadísticos, logran predecir eventos futuros basados en dichas relaciones.
Dado que ya se cuenta con los datos históricos de miles de pacientes, este tipo de tecnología podría servir perfectamente para resolver la problemática actual de este cliente. 
El desarrollo de este proyecto tendría dos resultados importantes: dotar a la organización de los conocimientos y herramientas necesarias para resolver problemas que requieran el uso de las nuevas tecnologías, y obtener un producto que resuelva una problemática real, ayudando al personal médico a guiar su trabajo sobre los pacientes más críticos y, en última instancia, salvar vidas.

\section{2. Identificación y análisis de los interesados}
\label{sec:interesados}

\begin{table}[ht]
%\caption{Identificación de los interesados}
%\label{tab:interesados}
\begin{tabularx}{\linewidth}{@{}|l|X|X|l|@{}}
\hline
\rowcolor[HTML]{C0C0C0} 
Rol           & Nombre y Apellido & Organización   & Puesto        \\ \hline
Cliente       & \begin{tabular}[c]{@{}l@{}}\clientename \\ Fabio Rosellini\end{tabular}      & \empclientename & Directores      \\ \hline
Responsable   & \authorname       & FIUBA           & Alumno        \\ \hline
Colaboradores & \begin{tabular}[c]{@{}l@{}}Colaborador 1 \\ Colaborador 2\end{tabular} &  Empresa médica  & \begin{tabular}[c]{@{}l@{}}Gerente de sistemas \\ Asesor de presidencia\end{tabular} \\ \hline
Orientador    & \supname          & \pertesupname    & Director de trabajo final \\ \hline
Usuario final & Personal médico y administrativo   & Empresa médica  & Usuario del sistema   \\ \hline
\end{tabularx}
\end{table}

\begin{itemize}
	\item Cliente: está a favor del desarrollo de este proyecto para impulsar la organización a nuevos mercados y/o ofrecerle desarrollos innovadores a los clientes actuales.
	\item Colaboradores: el asesor de presidencia no tiene mucho tiempo para dedicarle al proyecto, pero puede aportar conocimientos médicos para seleccionar las características más importantes que tengan relación con la mortalidad de los pacientes. Está a favor del desarrollo del proyecto y aportará lo que sea necesario. El gerente de sistemas tampoco tiene mucho tiempo para dedicarle al proyecto, pero es quien dará el permiso para la utilización de los datos. Como no solicitó el desarrollo del proyecto puede que demore en responder si se le consulta algo referido a la infraestructura.
	\item Orientador: puede ayudar mucho en el tratamiento de los datos antes de entrenar los modelos.
	\item Usuario final: necesitará que la predicciones que realiza el modelo estén disponibles en todo momento.
\end{itemize}


\section{3. Propósito del proyecto}
\label{sec:proposito}

El propósito principal de este proyecto es dotar a la organización de nuevas herramientas y capacidades en el ámbito de la inteligencia artificial y el análisis de datos, para poder ofrecer soluciones innovadoras a clientes nuevos o existentes, y poder insertarse en un mercado que se encuentra en constante crecimiento. 

\section{4. Alcance del proyecto}
\label{sec:alcance}

El presente proyecto incluye principalmente la instalación y configuración de una plataforma de gestión de modelos, que permita administrar versiones y ejecutar acciones como el despliegue de modelos.
Como prueba piloto, se desarrollará un modelo de predicción de mortalidad, que se incorporará a la plataforma de gestión, y también una interfaz que actúe como nexo entre el modelo predictivo y el usuario final.
Para el desarrollo del modelo, se realizarán las tareas de extracción de los datos, su procesamiento, el entrenamiento de los modelos y la evaluación de sus métricas. 
Una vez que se cuente con un modelo entrenado, se realizarán las tareas referidas al despliegue del modelo. 
Con el fin de automatizar la llamada al modelo, se desarrollará también un proceso que envíe automáticamente pedidos de predicciones de todos los pacientes activos cada cierto tiempo al modelo, para que el usuario final cuente con un reporte actualizado en todo momento.
El resultado del modelo podrá ser binario (hay riesgo o no hay riesgo), o categórico, devolviendo un grado de riesgo para ese paciente. Esto se definirá luego de realizar un análisis de los datos disponibles y de los modelos seleccionados para el entrenamiento.

El presente proyecto no incluye el desarrollo de una plataforma de gestión, sino que se elegirá una existente que cumpla con los requerimientos del proyecto. No se desarrollará una interfaz web orientada al usuario final, sino que se desarrollará un servicio web que sea utilizado por procesos automáticos. No se incluirá un proceso de re-entrenamiento automático del modelo.
No se incluye la instalación del modelo predictivo en el entorno productivo de la empresa médica, sino que para el proyecto presentado para este posgrado, se trabajará en un entorno propio de Grupo DUAM que incluirá la plataforma de gestión de modelos, el modelo predictivo y la interfaz por servicio web. Lo único que se instalará en el entorno de la empresa médica será el proceso que recupere datos de los pacientes y llame al servicio web cada cierto período de tiempo para recuperar las predicciones. Esto debe aclararse ya que la instalación en producción de toda la infraestructura no es el objetivo principal de este proyecto, y podría dejarse a cargo del equipo de sistemas de la empresa médica.

\section{5. Supuestos del proyecto}
\label{sec:supuestos}

Para el desarrollo del presente proyecto se supone que:

\begin{itemize}
	\item Se dispone de tiempo en horario laboral para avanzar en el proyecto.
	\item Se dispone de los equipos necesarios para realizar el procesamiento de los datos y el entrenamiento de los modelos.
	\item Se dispone de un ambiente donde poder instalar y configurar la plataforma de gestión, disponibilizar la interfaz por web service y desplegar el modelo para realizar predicciones. 
	\item No hay urgencia en el desarrollo del modelo predictivo.		
	\item Se tiene acceso a los datos médicos de los pacientes.
	\item Se cuenta con un conjunto inicial de datos lo suficientemente grande y representativo para entrenar los modelos de inteligencia artificial de manera efectiva.
	\item Se cuenta con el apoyo de un responsable médico que dé soporte en la selección de variables médicas a utilizar en el entrenamiento de los modelos y también en la interpretación clínica de las predicciones realizadas.
	\item Se mantendrá la protección de datos sensibles de los pacientes y de la empresa médica en todo momento.		
\end{itemize}

\section{6. Requerimientos}
\label{sec:requerimientos}

\begin{enumerate}
	\item Requerimientos funcionales
		\begin{enumerate}
			\item La plataforma de gestión de modelos deberá permitir desplegar modelos en diversos ambientes.
			\item La interfaz por servicio web deberá recibir datos médicos de uno o varios pacientes y devolver las predicciones asociadas a ellos.			
			\item El modelo predictivo deberá tener una precisión de al menos un 75\%.
			\item El proceso que solicita predicciones y genera el reporte al usuario deberá poder ejecutarse automáticamente cada cierto período de tiempo.		
			\item El reporte de predicciones que le llegue al usuario final deberá tener un formato claro y comprensible.
			\item Se utilizará GIT como repositorio para el control de version de código
		\end{enumerate}
	\item Requerimientos de datos a utilizar
		\begin{enumerate}		
		\item Durante el entrenamiento del modelo se deberá resguardar la confidencialidad de los datos de los pacientes.		
		\end{enumerate}
	\item Requerimientos de documentación
		\begin{enumerate}
			\item Redactar una memoria técnica con la información del proyecto.
			\item La documentación de la interfaz por servicio web deberá incluir la lista de métodos disponibles con su detalle.
			\item La documentación del modelo predictivo incluirá información sobre el origen de los datos utilizados para el entrenamiento, las características que se usaron, el detalle del modelo seleccionado y la información que haya sobre la explicabilidad del modelo.
		\end{enumerate}		
\end{enumerate}

\section{7. Historias de usuarios (\textit{Product backlog})}
\label{sec:backlog}

Para poder estimar la cantidad de puntos de historia que representa cada historia de usuario se usarán los siguientes criterios y grados:

\begin{itemize}
	\item Dificultad
	\begin{itemize}
		\item Baja: 1 punto
		\item Media: 3 puntos
		\item Alta: 5 puntos
	\end{itemize}
	\item Complejidad
	\begin{itemize}
		\item Baja: 2 puntos
		\item Media: 5 puntos
		\item Alta: 7 puntos
	\end{itemize}
	\item Incertidumbre
	\begin{itemize}
		\item Baja: 1 punto
		\item Media: 5 puntos
		\item Alta: 10 puntos
	\end{itemize}
\end{itemize}

El puntaje final de cada historia de usuario será calculado como el número de la secuencia de Fibonacci más próximo mayor o igual a la suma de las calificaciones en los 3 criterios.

\begin{itemize}
	\item Como director de la empresa Grupo DUAM, quiero contar con un ambiente en el cual se puedan incluir nuevos modelos predictivos que resuelvan las problemáticas de mis clientes, que sea sencilla y fácil de configurar. (3+5+10 = 18 → 21 puntos)
	\item Como desarrollador de la empresa Grupo DUAM, quiero poder reutilizar los componentes que se desarrollen en proyectos de otros clientes. (3+2+5 = 10 → 13 puntos)
	\item Como médico de centro de la empresa médica, quiero contar con un reporte actualizado de riesgos de mortalidad de los pacientes de mi centro, para poder realizar ajustes en los tratamientos y medicaciones prescritas. (1+5+1 = 7 → 8 puntos)
	\item Como gerente de sistemas de la empresa médica, quiero poder realizar el despliegue del modelo predictivo en producción de manera simple y sencilla. (3+5+1 = 9 → 13 puntos)
\end{itemize}

\section{8. Entregables principales del proyecto}
\label{sec:entregables}

Los entregables del proyecto son:

\begin{itemize}
	\item Documentación del servicio web.
	\item Documentación del modelo predictivo. 
	\item Modelo predictivo de mortalidad.
	\item Plataforma de gestión de modelos correctamente configurada.
	\item Interfaz por servicio web.
	\item Proceso que solicite predicciones y arme un reporte para el usuario de forma automática.
	\item Informe final.
\end{itemize}

\section{9. Desglose del trabajo en tareas}
\label{sec:wbs}

\begin{enumerate}
\item Desarrollo del modelo predictivo de mortalidad (280 h).
	\begin{enumerate}
	\item Investigación sobre variables médicas que tienen influencia en la mortalidad (20 h).
	\item Extracción de datos médicos de la base de datos (40 h).
	\item Procesamiento de datos médicos (40 h).
	\item Entrenamiento de modelos de \textit{machine learning} (40 h).
	\item Entrenamiento de modelos de \textit{deep learning} (40 h).
	\item Evaluación de métricas de los modelos (40 h).
	\item Selección de los modelos candidatos (20 h).
	\item Ajuste fino de los modelos seleccionados (40 h).	
	\end{enumerate}
\item Desarrollo de la interfaz por servicio web (80 h).
	\begin{enumerate}
	\item Diseño de la interfaz (20 h).
	\item Implementación de la interfaz (40 h).
	\item Pruebas sobre la interfaz para obtener predicciones (20 h).	
	\end{enumerate}
\item Instalación y configuración de la plataforma de gestión de modelos (60 h).
	\begin{enumerate}
	\item Investigación de plataformas de gestión de modelos disponibles (10 h).
	\item Instalación de plataforma de gestión seleccionada (30 h).
	\item Configuración de ambientes para despliegue de modelos (20 h).
	\end{enumerate}
\item Desarrollo del proceso automático para solicitar predicciones y generar reportes (80 h).
	\begin{enumerate}
	\item Diseño del proceso que solicita predicciones automáticamente (20 h).
	\item Desarrollo del proceso que solicita predicciones automáticamente (40 h).
	\item Pruebas de automatización de llamadas al modelo (20 h).
	\end{enumerate}
\item Documentación (100 h).
	\begin{enumerate}
	\item Documentación del servicio web (20 h).
	\item Documentación del modelo predictivo (30 h).
	\item Documentación de informe de avance (20 h).
	\item Informe final del proyecto (30 h).
	\end{enumerate}
\end{enumerate}

Cantidad total de horas: 600 h

\section{10. Diagrama de Activity On Node}
\label{sec:AoN}

A continuación, se realiza una breve descripción del Diagrama de Activity On Node que se muestra en la figura \ref{fig:AoN}:

\begin{itemize}
	\item Las tareas de desarrollo del modelo predictivo son las primeras que deben realizarse, ya que de ellas dependen otras actividades como el desarrollo de la interfaz o el desarrollo del proceso automático. 
	\item Las tareas de instalación y configuración de la plataforma de gestión de modelos puede iniciarse junto con el desarrollo del modelo, pero no podrá finalizar hasta que se cuente con el modelo final.
	\item A medida que las tareas de desarrollo vayan finalizado, se documentarán los productos terminados.
\end{itemize}

\begin{figure}[htpb]
\centering 
\includegraphics[width=0.99\textwidth]{./Figuras/AoN_Extendido.pdf}
\caption{Diagrama de \textit{Activity on Node} con su camino crítico marcado en negrita.}
\label{fig:AoN}
\end{figure}

El camino crítico muestra que como mínimo se requieren 450 horas para finalizar el proyecto.


\section{11. Diagrama de Gantt}
\label{sec:gantt}

En la figura \ref{fig:diagGantt01}, se muestra la tabla del Diagrama de Gantt y en la figura \ref{fig:diagGantt02} se muestran las tareas desplegadas sobre la línea de tiempo.

\begin{figure}[htpb]
\centering 
\includegraphics[height=.8\textheight]{./Figuras/Gantt-1.png}
\caption{Tabla del Diagrama de Gantt.}
\label{fig:diagGantt01}
\end{figure}


\begin{landscape}
\begin{figure}[htpb]
\centering 
\includegraphics[height=.9\textheight]{./Figuras/Gantt-2.png}
\caption{Diagrama de Gantt.}
\label{fig:diagGantt02}
\end{figure}

\end{landscape}

\section{12. Presupuesto detallado del proyecto}
\label{sec:presupuesto}

A continuación, se detalla la composición del presupuesto del proyecto expresado en pesos argentinos:

\begin{table}[htpb]
\centering
\begin{tabularx}{\linewidth}{@{}|X|c|r|r|@{}}
\hline
\rowcolor[HTML]{C0C0C0} 
\multicolumn{4}{|c|}{\cellcolor[HTML]{C0C0C0}COSTOS DIRECTOS} \\ \hline
\rowcolor[HTML]{C0C0C0} 
Descripción &
  \multicolumn{1}{c|}{\cellcolor[HTML]{C0C0C0}Cantidad} &
  \multicolumn{1}{c|}{\cellcolor[HTML]{C0C0C0}Valor unitario} &
  \multicolumn{1}{c|}{\cellcolor[HTML]{C0C0C0}Valor total} \\ \hline
  
 Mano de obra &
  \multicolumn{1}{c|}{600 h} &
  \multicolumn{1}{c|}{\$4.600/h} &
  \multicolumn{1}{c|}{\$2.760.000} \\ \hline
\multicolumn{3}{|c|}{SUBTOTAL} &
  \multicolumn{1}{c|}{\$2.760.000} \\ \hline
\rowcolor[HTML]{C0C0C0} 
\multicolumn{4}{|c|}{\cellcolor[HTML]{C0C0C0}COSTOS INDIRECTOS} \\ \hline
\rowcolor[HTML]{C0C0C0} 
Descripción &
  \multicolumn{1}{c|}{\cellcolor[HTML]{C0C0C0}Cantidad} &
  \multicolumn{1}{c|}{\cellcolor[HTML]{C0C0C0}Valor unitario} &
  \multicolumn{1}{c|}{\cellcolor[HTML]{C0C0C0}Valor total} \\ \hline
\multicolumn{1}{|l|}{40 \% Costos indirectos} &
 \multicolumn{1}{c|}{600 h} &
  \multicolumn{1}{c|}{\$1.840/h} &
  \multicolumn{1}{c|}{\$1.104.000} \\ \hline
\multicolumn{3}{|c|}{SUBTOTAL} &
  \multicolumn{1}{c|}{\$1.104.000} \\ \hline
\rowcolor[HTML]{C0C0C0}
\multicolumn{3}{|c|}{TOTAL} & \$3.864.000
   \\ \hline
\end{tabularx}%
\end{table}


\section{13. Gestión de riesgos}
\label{sec:riesgos}

a) Para la estimación de severidad y probabilidad de ocurrencia de los riesgos del trabajo se utilizó una escala de 1 a 10, donde 10 es el valor máximo posible para cada índice. Dentro de los riesgos detectados se encuentran:

Riesgo 1: no disponer de un conjunto de datos adecuado para el entrenamiento del modelo.
\begin{itemize}
	\item Severidad (S): este riesgo tiene una severidad alta, ya que si no se cuenta con un conjunto de datos lo suficientemente grande, el modelo no podrá generalizar el conocimiento y realizar predicciones precisas. (9)
	\item Probabilidad de ocurrencia (O): la probabilidad de que el riesgo suceda es baja, ya que la empresa médica registra datos de pacientes desde hace mas de 10 años y cuentan con una base de datos de cerca de 10.000 pacientes. (3)
\end{itemize}   

Riesgo 2: no lograr que el modelo entrenado realice predicciones correctas.
\begin{itemize}
	\item Severidad (S): este riesgo tiene una severidad alta, ya que si el modelo no realiza predicciones correctas el usuario dejará de confiar en la información generada y rechazará el producto. (10)
	\item Probabilidad de ocurrencia (O): la probabilidad de que el riesgo suceda es baja, ya que existen muchas técnicas para procesar los datos y muchos modelos con estructuras distintas para entrenar, de cuyas combinaciones se espera encontrar una que genere buenas predicciones. (4)
\end{itemize} 

Riesgo 3: falta de colaboración por parte de los médicos interesados para la selección de variables que tengan relación con la mortalidad.
\begin{itemize}
	\item Severidad (S): este riesgo tiene una severidad baja, ya que aunque no se cuente con asesoramiento médico para la selección de características del conjunto de datos, se puede abordar el problema desde el punto de vista estadístico, aplicando las técnicas de procesamiento de datos adecuada para cada variable. El proceso de entrenamiento requerirá mas tiempo, ya que se deberán hacer mas pruebas para encontrar el conjunto de características que ayuden al modelo a generalizar el conocimiento y generar predicciones correctas. (4)
	\item Probabilidad de ocurrencia (O): la probabilidad de que el riesgo suceda es media. Si bien el asesor de presidencia, principal interesado en el modelo predictivo, dispone de poco tiempo para el proyecto, se puede mantener un contacto asincrónico para ayudar a que el proyecto avance sin ocupar demasiado tiempo. (5)
\end{itemize} 

Riesgo 4: pérdida o daño en los archivos del proyecto.
\begin{itemize}
	\item Severidad (S): este riesgo tiene una severidad alta, ya que si se pierden o dañan los archivos del proyecto habrá que comenzarlo de nuevo, perdiendo gran parte del trabajo realizado. (8)
	\item Probabilidad de ocurrencia (O): la probabilidad de que el riesgo suceda es baja, ya que los archivos del proyecto se pueden subir en repositorios remotos y mantener copias de seguridad para evitar su perdida o daño. (2)
\end{itemize} 

Riesgo 5: no finalizar las tareas según las fechas planificadas.
\begin{itemize}
	\item Severidad (S): este riesgo tiene una severidad baja, ya que mientras se cumpla el objetivo principal del proyecto se puede admitir una demora en la finalización del proyecto. (1)
	\item Probabilidad de ocurrencia (O): la probabilidad de que el riesgo suceda es media, ya que es el primer proyecto que se desarrolla utilizando estas tecnologías y pueden existir algunas demoras que extiendan los plazos definidos. (5)
\end{itemize}

b) Tabla de gestión de riesgos:      (El RPN se calcula como RPN=SxO)

\begin{table}[htpb]
\centering
\begin{tabularx}{\linewidth}{@{}|X|c|c|c|c|c|c|@{}}
\hline
\rowcolor[HTML]{C0C0C0} 
Riesgo & S & O & RPN & S* & O* & RPN* \\ \hline
No disponer de un conjunto de datos adecuado para el entrenamiento del modelo & 9 & 3 & 27 & 9* & 2* & 18* \\ \hline
No lograr que el modelo entrenado realice predicciones correctas & 10 & 4 & 40 & 10* & 2* & 20* \\ \hline
Falta de colaboración por parte de los médicos interesados para la selección de variables que tengan relación con la mortalidad        & 4 & 5 & 20 & - & - & - \\ \hline
Pérdida o daño en los archivos del proyecto & 8 & 2 & 16 & - & - & - \\ \hline
No finalizar las tareas según las fechas planificadas & 1  & 5 & 5 & - & - & - \\ \hline
\end{tabularx}%
\end{table}

Criterio adoptado: 
Se tomarán medidas de mitigación en los riesgos cuyos números de RPN sean mayores a 20.

Nota: los valores con (*) en la tabla representan los valores luego de haber aplicado la mitigación.

c) Plan de mitigación de los riesgos que originalmente excedían el RPN máximo establecido:

Riesgo 1. No disponer de un conjunto de datos adecuado para el entrenamiento del modelo: el plan de mitigación para este riesgo consiste en aumentar la cantidad de datos disponibles con una técnica conocida como \textit{"data augmentation"}, donde se generan nuevos datos a partir de los datos existentes para ayudar al modelo a generalizar el conocimiento.
\begin{itemize}
	\item Severidad (S): la severidad se mantiene alta, ya que si se aplican las técnicas de aumento de datos y aún así no se tienen datos suficientes para entrenar al modelo, no se lograrían predicciones correctas. (9)
	\item Probabilidad de ocurrencia (O): Baja levemente la probabilidad de ocurrencia de este riesgo, ya que la técnica aplicada para mitigar el riesgo suele usarse para estos casos y brinda buenos resultados.(2)
\end{itemize}

Riesgo 2. No lograr que el modelo entrenado realice predicciones correctas: el plan de mitigación para este riesgo consiste en investigar que modelos suelen ser mejores para resolver este tipo de problemas, trabajar conjuntamente con personal médico para seleccionar las características mas importantes del conjunto de datos y dedicarle tiempo extra a los entrenamientos de los modelos con el fin de lograr buenos resultados.
\begin{itemize}
	\item Severidad (S): la severidad se mantiene alta, ya que si se elijen los modelos mas apropiados, se seleccionan las características mas importantes y se dedica tiempo extra sin obtener buenos resultados, el modelo no será útil. (10)
	\item Probabilidad de ocurrencia (O): Baja levemente la probabilidad de ocurrencia de este riesgo, ya que si se busca la información correcta, se investigan resoluciones de casos similares, y se dedica mas tiempo al entrenamiento del modelo, es probable que se consigan mejores resultados.(2)
\end{itemize}

\section{14. Gestión de la calidad}
\label{sec:calidad}

\begin{itemize}
	\item Req \#1.1: La plataforma de gestión de modelos deberá permitir desplegar modelos en diversos ambientes.
	\begin{itemize}
		\item Verificación: verificar la configuración de la plataforma de despliegue para comprobar que se toma la última versión del modelo y se despliega en el ambiente seleccionado.
		\item Validación: ingresar a la plataforma con un usuario y contraseña y hacer clic en un botón que despliegue el modelo en el ambiente configurado. 
	\end{itemize}
	\item Req \#1.2: La interfaz por servicio web deberá recibir datos médicos de uno o varios pacientes y devolver las predicciones asociadas a ellos.	
	\begin{itemize}
		\item Verificación: realizar pruebas de llamada a la interfaz por servicio web, enviando datos de uno o varios pacientes, y recibir las predicciones.
		\item Validación: ejecutar el proceso automático que genera las predicciones de todos los pacientes activos y mostrar a los usuarios de la empresa médica el reporte generado.
	\end{itemize}			
	\item Req \#1.3: El modelo predictivo deberá tener una precisión de al menos un 75\%.
	\begin{itemize}
		\item Verificación: realizar las inferencias del set de datos de \textit{test} y verificar que la métrica "precisión" del modelo se encuentre dentro de los valores aceptables.
		\item Validación: realizar un gráfico con los valores obtenidos para cada métrica y validar que la métrica "precisión" del modelo se encuentre dentro de los valores aceptables.
	\end{itemize}
	\item Req \#1.4: El proceso que solicita predicciones y genera el reporte al usuario deberá poder ejecutarse automáticamente cada cierto período de tiempo.		
	\begin{itemize}
		\item Verificación: configurar el proceso automático para que se ejecute cada un minuto, esperar dicho tiempo y verificar que se haya ejecutado el proceso y generado el reporte.
		\item Validación: esperar a que el proceso se ejecute automáticamente en el tiempo establecido y validar que se genera el reporte correctamente.
	\end{itemize}
	\item Req \#1.5: El reporte de predicciones que le llegue al usuario final deberá tener un formato claro y comprensible.
	\begin{itemize}
		\item Verificación: ejecutar el proceso automático que genera el reporte de predicciones y verificar que se pueda abrir el archivo y tenga un formato correcto.
		\item Validación: generar el reporte de predicciones y enviar a los usuarios de la empresa médica para que validen si el formato es correcto.
	\end{itemize}
	\item Req \#1.6: Se utilizará GIT como repositorio para el control de version de código.
	\begin{itemize}
		\item Verificación: revisar repositorio en la nube para verificar que se encuentra subida la última versión del código fuente.
		\item Validación: compartir acceso al repositorio al gerente de sistemas de la empresa médica para visualizar el código fuente.
	\end{itemize}	
	\item Req \#2.1: Durante el entrenamiento del modelo se deberá resguardar la confidencialidad de los datos de los pacientes.	
	\begin{itemize}
		\item Verificación: almacenar los datos únicamente en directorios privados de Grupo DUAM o de la empresa médica.
		\item Validación: compartir ubicación de los datos solamente al equipo de desarrollo de Grupo DUAM y al equipo de sistemas de la empresa médica.
	\end{itemize}
	\item Req \#3.1: Redactar una memoria técnica con la información del proyecto.
	\begin{itemize}
		\item Verificación: escribir una memoria a medida que se avanza con el proyecto.
		\item Validación: compartir la memoria técnica con la organización Grupo DUAM.
	\end{itemize}
	\item Req \#3.2: La documentación de la interfaz por servicio web deberá incluir la lista de métodos disponibles con su detalle.
	\begin{itemize}
		\item Verificación: escribir documento que incluya el detalle de cada método desarrollado una vez que se desarrolle la interfaz.
		\item Validación: compartir la documentación de la interfaz con la organización Grupo DUAM y con la empresa médica.
	\end{itemize}
	\item Req \#3.3: La documentación del modelo predictivo incluirá información sobre el origen de los datos utilizados para el entrenamiento, las características que se usaron, el detalle del modelo seleccionado y la información que haya sobre la explicabilidad del modelo.
	\begin{itemize}
		\item Verificación: escribir documentación con detalles del proceso de entrenamiento y características del modelo seleccionado.
		\item Validación: compartir documentación del modelo predictivo con la organización Grupo DUAM y con la empresa médica.
	\end{itemize}
\end{itemize}

\section{15. Procesos de cierre}    
\label{sec:cierre}

\begin{itemize}
	\item Reunión por videollamada con el director del proyecto con el fin de analizar las conclusiones sobre el resultado final del proyecto y las lecciones aprendidas para tener en cuenta en trabajos posteriores.
	\item El responsable del proyecto será quien se ocupe de redactar una minuta de la reunión por videollamada, así como también tomar nota en una planilla sobre que herramientas y procesos fueron útiles, cuáles no y el motivo.
	\item Se presentará el trabajo final por el \authorname\hspace{1px} donde se agradecerá a todas las personas involucradas en el proyecto, miembros del jurado, docentes y autoridades de la CEIA.
\end{itemize}


\end{document}
